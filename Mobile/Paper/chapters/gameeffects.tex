Multiple visual effects can be found in Space Shooter. The most notable one is the force field effect around the player ship. This shield is in fact a sperical particle system, with particles projected on a virtual sphere to give it an animated look. A transparent material is applied to this sphere, and randomly generated noise that provides the blurring effect across a slightly colored hue. The particle system has a pulse effect, with a frequency of roughly 0.7 seconds caused by the particle emission. This phenomena occurs because that's how long a particle emission cycle lasts in this particular effect. The particle system will continuously loop, giving the impression of a continuous bubble, which only stops when the player's shield points reach 0. Once they regenerate sufficiently, the shield will reactivate along with its particle system. \\
The shield has an additional visual effect, upon collision with other objects. The collision triggers a new shader effect, which sends additional noise across the spherical surface, as a shockwave ring moving away from the point of impact. \\ \\
Another effect is the engine trail on the player ship. It is based on Unity's Trail Renderer, which functions similarly to a particle system, except instead of particles being constantly thrown out, they are left behind as the object is moved around the game world. The pefromance impact of trail renderers is quite small, but they are often paired with Point Ligth objects when creating propulsion effects. A similar effect, without extra light, is used for the player weapon projectiles, to facilitate lighting while making the game a little more spectacular.\\
 Light sources in the game can be a dangerous tool. They bring life to a game scene, but, depending on configuration, they can be costly to process. Like everything in a game, light is not real, and must be simulated. But depending on the choice of rendering, and of desired fidelity, lighting can be one of the harshest things to process in a game. In this case, it is a single point light that only casts very simple shadows. \\ \\ 
 The last major effect to be noted in the game is the explosion. Whenever something in Space Shooter is destroyed, it explodes, whether it is an asteroid or a drone. They are classical one-shot particle systems, using many, fiery colored particles that launch in a quick, expanding, burst, and imploding afterwards. This effect can potentially bring high performance penalties to the game, depending on how it is implemented. \\ \\
 As a conclusion regarding Space Shooter, there are a lot of basic tools present in Unity, basic but powerful, with which a content creator can make fascinating effects. In the next secions we will discuss the real-world aspects of game development, how to measure a game's performance and then improve it.