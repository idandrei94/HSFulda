Unity is among the world's leading real-time creation platform. It supports over 25 different platform, developers being able to reach a vast variety of audience, some examples being the 3 major PC operating systems (Windows, Mac and Linux), mobile platforms (iOS, Android), as well as gaming platforms (Playstation, Xbox One). \\
The Unity3d game engine supports a variety of genres, providing both 2D and 3D engines, physics, AI, pathfinding and Virtual/Augmented Reality systems such as Google ARCore, Apple ARKit and Vuforia. \cite{WhatIsUnity} \\ \\
Unlike other examples of software development, the borders of "acceptable performance" are not so well defined for games. Many applications are designed for well-determined scenarios, platforms and specifications, and improvements in performance past those specifications become unnecessary investments. When developing a game however, even if performance is acceptable on a more powerful platform, with a fraction of the development cost, the game can be optimized further and improve the player's experience or open the possibility of expanding the audience and thus revenue. \\ \\
A prime, relatively recent example is the game Fortnite, developed by Epic Games. Announced formally in September 2011, the game's popularity increased drastically while still early-access, until March 2018, when the developers announced cross-platform support coming along the game's release on the iOS mobile platform. By august 2018, the game was available also on Android, Xbox, Nintendo Switch and PlayStation 4. \cite{Fortnite}. The large availability spiked the game's popularity even further, enabling gamers everywhere to have their favorite game on the go, not requiring expensive investments to play. \\ \\
This paper will focus on an analysis over various techniques aimed at improving a game's performance. The study will be performed over a sample game developed using Unity 3d, which will be the development platform of choice. The study will also showcase cross-platform development of a game, as testing will be done on both a Windows PC and an Android mobile device. It is recommended that the reader familiarizes themselves with basic video game terminology before reading. Knowledge of Object-Oriented programming is assumed.