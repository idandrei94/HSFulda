Video games can be very computationally intensive. Certain tasks rely on the machine's CPU power (script execution, physics, game object search/creation), others on the GPU (lighting, texture processing, camera rendering). Different platforms have different specs, with great possible deviations from the average. The processing power of some devices can be strongly imbalanced towards the CPU, providing only minimal support for graphics. \\ \\
Unity provides great resource usage data to track which part of the hardware is most taxed, which is often the best place to start optimizing. The Breakdown View can be used to see the call stacks on the CPU and GPU, see which tasks take the most, keep track of memory usage and thus identify which elements have the highest negative impact. \\ \\
Most often, the cause of low performance is suboptimal usage of the CPU and/or system memory by the MonoBehaviour, the base class for any script attached to game objects, or the physics/collision systems. As far as the GPU is concerned, prime exampled would be the needless usage of high-resolution texture, too many/intense particle systems or the use of high-fidelity rendering techniques such as ray tracing. The target rendering resolution also plays an important role as far as GPU loads are concerned.