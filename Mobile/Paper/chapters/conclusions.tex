In this paper we have looked at several means of improving and stabilizing the performance of a sample game implemented in Unity. While the end results may not be called extreme, significant improvements can be seen between the initial and final reading as per table 5. Looking again at the first and last values, using a few basic techniques, Space Shooter has gained more than doubled the average framerate on PC with extremely similar deviations, and the Android version was improved by over 91\%, albeit with more fluctuations. \\
Looking at the rest of the entries, we can see that the previous statements hold true, with extremely similar changes between the original and optimized version of the game.
``The one constant cost included in all performance optimization work is time. So, with
limited time at our disposal to both implement the features we want to implement and keep
everything working, an important skill to learn for any developer is workflow optimization.
Better understanding of the tools we use will save us more time in the long run, and
hopefully provide the extra time we need to implement everything we want to, which
applies not only to the Unity Engine, but to every tool we use.''\cite{optimizationbook}
\begin{table}
\caption{Optimization data}
\label{tab:conf}
\begin{minipage}{0.49\textwidth}
\begin{center}
\begin{tabular}{lllll}
Value & PC pre & PC post & Android pre & Android post \\
Mean & 75.00 & 155.04 & 10.30 & 19.70 \\
Std & 2.51 & 2.77 & 0.67 & 7.12 \\
Top 10 mean & 76.00 & 159.20 & 12.30 & 28.60 \\
Bottom 10 mean & 67.50 & 150.30 & 10.00 & 10.00 \\
Min & 62 & 146 & 10 & 10 \\
Max & 76 & 160 & 13 & 29 
\end{tabular}
\bigskip
\end{center} 
\end{minipage}
\end{table}