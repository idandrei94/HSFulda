Performance tests determine how fast the system or some parts of the system can work under the particular workload. Load testing, stress testing, configuration testing, spike testing are some of the sub categorization of performance testing. \cite{testingarticle} \\ \\
Unlike specialized software, a game can greatly benefit from its availability on as many devices as possible, thus facilitating the highest player base possible. As an example, many productivity tools could prove a cumbersome experience when used on a smartphone, such as Autodesk products, while a game available on mobile devices enable users to play it everywhere, providedthat suitable means of input are implemented. According to the Cambridge Dictionary of English, a game is defined as `` an entertaining activity, esp. one played by children, or a sports competition'' \cite{gamedef}. As an entertinament source, games cannot therefore be considered critical applications, providing great flexibility in regard to specifications. \\ \\
Performance analysis in games may not exclusively be used to prove that the product complies with specifications, but also to identify headroom. If performance is way above expectations, the game could be a candidate for availability on lower performance platforms that were not previously considered. Unity's cross-platform development solution makes this relatively easy by allowing the developer to create platform specific quality settings, as will be discussed later on in this paper.\\ \\
When the gameplay experience is not stable and constant, measures must be taken to address this issue. When tracking the game performance during development, a consistent test environment is needed to ensure results are not biased by external factors (a more powerful machine may yield better results despite a later version actually performing worse than the previous one tested on a slower machine). Thus, when preparing for tests, all the teams should update their code, tests, documents and test environment and align them \cite{gamedef} to ensure consistent results.
The following subsections will present the methodology and environments used to test Space Shooter. Being a purely demonstrative application for key consepts, the game has no product documentation, nor does it have specifications. Testing will be focused only on the performance of the existing application, in its current form, and the changes caused by the implementation of the techniques discussed in future sections.