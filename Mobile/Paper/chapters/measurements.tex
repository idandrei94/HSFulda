Like other forms of computer graphics, the game is rendered as a rapid succession of images, rapid enough that the human eye perceives continuous movement. These images are called framerates, and the speed at which they are displayed is referred to as framerate, measured in a second, and called Frames Per Second or FPS. \\
According to a York University research in partnership with others, in all conditions tested, there was a clear preference for higher frame rates (48fps and 60fps) when contrasted with a standard of 24fps, regardless of content. \cite{fpspaper}. And while the transition from 48 to 60 FPS was not always showing improvement in general, the study also shows that, when dealing with high-action scenes with rapid movement, the preservation of the illusion of movement offered by increased framerates has a noticeable impact on the viewer. \\ \\
Given these facts, the framerate proves to be a reliable and relevant means of measuring a game's performance across various platforms. An initial goal of 24 FPS will be aimed for, and for a hypothetical mobile release, the framerate will be limited to this value to preserve battery life in a mobile device. For the purpose of testing however, we will attempt to reach higher values if possible. \\
Given that mobile devices are not meant for multitasking, we will not be too concerned about resource usage otherwise, provided the device has sufficient resources to run the game in the first place. Should this not be the case, the framerate will be considered zero.